\documentclass[11pt, oneside]{article}
\usepackage{geometry}                		% See geometry.pdf to learn the layout options. There are lots.
\geometry{letterpaper}                   		% ... or a4paper or a5paper or ... 
\usepackage[parfill]{parskip}    			% Activate to begin paragraphs with an empty line rather than an indent
\usepackage{graphicx}
\usepackage{amssymb}
\usepackage{mathtools}
\usepackage{enumerate}
\usepackage{tikz}
\usepackage{pgfplots}
\usepackage{hyperref}
\usepackage{mathtools}
\hypersetup{
	colorlinks=true,
	linkcolor=blue,
	filecolor=magenta,      
	urlcolor=cyan,
	pdftitle={Overleaf Example},
	pdfpagemode=FullScreen,
}
\pgfplotsset{width=10cm,compat=1.9}

\usetikzlibrary{arrows}

\def\firstcircle{(90:1.75cm) circle (2.5cm)}
\def\secondcircle{(210:1.75cm) circle (2.5cm)}
\def\thirdcircle{(330:1.75cm) circle (2.5cm)}
\graphicspath{ {img/} }
\newcommand{\E}{\mathbb{E}}
\usepackage{bbm}
\usepackage{caption,subcaption}
\renewcommand{\r}{\rho}
\renewcommand{\d}{\delta}
\renewcommand{\b}{\beta}

%SetFonts


\title{Macroeconomics-2 - Assignment \#2}
\author{Daniil Buchko - \texttt{dbuchko@nes.ru}}
\date{November 17, 2021}

\begin{document}
	
\maketitle

% ================= QUESTION 1  =================
\section*{Question 1}
\fbox{\begin{minipage}{40em}
Assume an individual who lives for infinite periods in a discrete time model. He has concave utility, earns an income of $ y $ each period. The subjective discount factor is equal to $ \beta =  1 / (1+\rho) $
and real interest rate is equal to $ r $. Draw the consumption path of this individual for each of the following cases:
\begin{enumerate}[(1)]
	\item $ r_{1} = r_{2} = r_{3} = \dots = \rho $
	\item $ r_{1} = r_{2} = r_{3} = \dots < \rho $
	\item $ r_{1} < \rho  $ and $ r_{2} = r_{3} = \dots = \rho $
	\item $ r_{1} = r_{5} = r_{6} = \dots = \rho  $ and $ r_{2} = r_{3} = r_{4} < \rho $
	\item $ r_{1} = r_{3} = r_{5} = \dots < \rho  $ and $ r_{2} = r_{4} = r_{6} = \dots = \rho  $
\end{enumerate}
\end{minipage}}

Before solving every point of the task, lets derive the maximization task and its solution. Individual solves the following task, by choosing values of $ c_{t} $:
\[
\begin{dcases}
	U(c) = \sum_{t=1}^{\infty} \frac{u(c_{t})}{(1+\r)^{t-1}} \to \max_{\{c_{t}\}_{t=1}^{\infty}}  \\
	\sum_{t=1}^{\infty}  \frac{c_{t}}{\prod_{i=1}^{t-1} (1+r_{i})} = 	\sum_{t=1}^{\infty}  \frac{y_{t}}{\prod_{i=1}^{t-1} (1+r_{i})}
\end{dcases}
\]
Lagrangian is the following:
\[
	\mathcal{L} = \sum_{t=1}^{\infty} \frac{u(c_{t})}{(1+\r)^{t-1}} + \lambda \left( \sum_{t=1}^{\infty}  \frac{y_{t}}{\prod_{i=1}^{t-1} (1+r_{i})} -  	\sum_{t=1}^{\infty}  \frac{c_{t}}{\prod_{i=1}^{t-1} (1+r_{i})}\right)
\]
Taking derivatives yields:
\[
	\begin{dcases}
		\frac{\partial \mathcal{L}}{\partial c_{t}} = 0: \quad \frac{u'(c_{t})}{(1+\r)^{t-1}} =  \frac{\lambda}{\prod_{i=1}^{t-1} (1+r_{i})} \quad\quad \forall t = 1\dots \\
		\frac{\partial \mathcal{L}}{\partial c_{t+1}} = 0: \quad \frac{u'(c_{t+1})}{(1+\r)^{t}} =  \frac{\lambda}{\prod_{i=1}^{t} (1+r_{i})} \quad\quad \forall t = 1\dots \\
	\end{dcases}
\]
Which is the \textbf{EE}:
\[
	 \frac{u'(c_{t})}{u'(c_{t+1})} = \frac{1+r_{t}}{1 + \r} \quad\quad \forall t = 1\dots
\]
\begin{enumerate}
	\item $ r_{1} = r_{2} = r_{3} = \dots = \rho$ 
		\[
			r_{t} = \r \implies u'(c_{t}) = u'(c_{t+1}) \implies c_{t} = c_{t+1} \quad\forall t = 1\dots
		\]

	\item $ r_{1} = r_{2} = r_{3} = \dots < \rho$
		\[
			r_{t} < \r \implies u'(c_{t}) < u'(c_{t+1}) 
		\]
	Considering the concavity of utility function we get:
		\[
			c_{t} > c_{t+1} \quad\forall t = 1 \dots
		\]
	\item $ r_{1} < \rho  $ and $ r_{2} = r_{3} = \dots = \rho $
		\[
			r_{1} < \r \implies u'(c_{1}) < u'(c_{2}) \implies c_{1} > c_{2} = c_{3} = \dots
		\]
	\item $ r_{1} = r_{5} = r_{6} = \dots = \rho  $ and $ r_{2} = r_{3} = r_{4} < \rho $
		\[
			\begin{dcases}
				r_{2} = r_{3} = r_{4} < \r \\
				r_{5} = r_{6} = \dots
			\end{dcases} \implies c_{2} > c_{3} > c_{4} > c_{5} = c_{6} = \dots
		\]
		and also applying that $ r_{1} = \r $:
		\[
			c_{1} = c_{2} > c_{3} > c_{4} > c_{5} = c_{6} = \dots
		\]
	\item $ r_{1} = r_{3} = r_{5} = \dots < \rho  $ and $ r_{2} = r_{4} = r_{6} = \dots = \rho  $
	\[
	\begin{dcases}
		u'(c_{2t - 1}) < u'(c_{2t}) \\ 
		u'(c_{2t}) = u'(c_{2t+1})
	\end{dcases} \implies c_{1}>c_{2} = c_{3} > c_{4} = c_{5} > \dots
	\]
	
\end{enumerate}
% ================= END QUESTION 1  =================

\clearpage
\section*{Question 4}
\fbox{\begin{minipage}{40em}
		Consider two-period model of consumption. Let the utility function of an individual be
\end{minipage}}
\begin{enumerate}[(1)] 
	\item $ y_{1} = 20, y_{2} = 90 $. Borrowing at $ r_{b} = 150\%$, saving at $ 50\% $.
	In order to solve the consumer problem we should find the maximum of utility function for two cases and compare their values:
	\[
	\begin{dcases}
		\max \sqrt{c_{1}} + \sqrt{c_{2}} \\
		c_{1} + \frac{c_{2}}{1+1.5} = 20 + \frac{90}{1+1.5}, \quad \text{ if } c_{1} \ge 20 \\
		c_{1} + \frac{c_{2}}{1+0.5} = 20 + \frac{90}{1+0.5}, \quad \text{ if } c_{1} \le 20
	\end{dcases} 
	\]
	\begin{enumerate}[(a)]
		\item Case $ c_{1} \ge Y_{1} $
		\[
		\mathcal{L} = \sqrt{c_{1}} + \sqrt{c_{2}} + \lambda_{1} \left( 20 + \frac{90}{2.5} - c_{1} - \frac{c_{2}}{2.5}  \right) + \lambda_{2}(c_{1} - 20)
		\]
		FOCs:
		\[
		\begin{dcases}
			\frac{\partial \mathcal{L}}{\partial c_{1}} = 0: \quad \frac{1}{2\sqrt{c_{1}}} - \lambda_{1} + \lambda_{2} = 0
			\\
			\frac{\partial \mathcal{L}}{\partial c_{2}} = 0: \quad  
			\frac{1}{2\sqrt{c_{2}}} - \lambda_{1} / 2.5 = 0 \\
			\lambda_{2}[c_{1} - 20] = 0
		\end{dcases} \text{ assuming } c_{1} > 20\implies \sqrt{\frac{c_{2}}{c_{1}}} = 2.5 \implies c_{2} = 6.25c_{1}
		\]
		Inserting to the budget constraint yields:
		\[
		3.5c_{1} = 56 \implies c_{1} = 16 < Y_{1} = 20
		\]
		We got $ c_{1} = 16 $, assuming that $ c_{1} > 20 $, therefore what we got is not a maximum. The only case left is $ c_{1} = 20  \implies c_{2} = 90$. Value of utility function at the maximum is $ U(20, 90)  = \sqrt{20} + \sqrt{90}$.
		\item Case $ c_{1} \le Y_{1} $:
		FOCs:
		\[
		\begin{dcases}
			\frac{\partial \mathcal{L}}{\partial c_{1}} = 0: \quad \frac{1}{2\sqrt{c_{1}}} - \lambda_{1} - \lambda_{2}  = 0
			\\
			\frac{\partial \mathcal{L}}{\partial c_{2}} = 0: \quad  
			\frac{1}{2\sqrt{c_{2}}} - \lambda_{1} / 1.5 = 0
		\end{dcases}  \text{ assuming } c_{1} < 20 \implies \sqrt{\frac{c_{2}}{c_{1}}} = 1.5 \implies c_{2} = 2.25c_{1}
		\]
		Inserting to the budget constraint yields:
		\[
		2.5c_{1} = 80  \implies c_{1} = 32 > Y_{1} = 20
		\]
		What we got is not a maximum. The only thing is left is $ c_{1} = 20 $, $ c_{2} = 90 $.
	\end{enumerate}
\item $ r_{b} = r_{s} = 0.5 $.
	\[
\begin{dcases}
	\max \sqrt{c_{1}} + \sqrt{c_{2}} \\
	c_{1} + \frac{c_{2}}{1+1.5} = 56
\end{dcases} 
\]
\[
\mathcal{L} = \sqrt{c_{1}} + \sqrt{c_{2}} + \lambda \left( 56 - c_{1} - \frac{c_{2}}{2.5} \right)
\]
\[
	\begin{dcases}
			\frac{\partial \mathcal{L}}{\partial c_{1}}:\quad \frac{1}{2\sqrt{c_{1}}} = \lambda \\
			\frac{\partial \mathcal{L}}{\partial c_{2}}:\quad \frac{1}{2\sqrt{c_{2}}} = \frac{\lambda}{2.5}
	\end{dcases} \implies c_{2} = 6.25 c_{1} \implies c_{2} = 100, c_{1} = 16, \quad s = Y_{1} - c_{1} = 4
\]
\item $ r_{b} = r_{s} = 0.5, y_{1} = 30, y_{2} = 90$.
Using results from the previous question $ c_{2} = 2.25 c_{1} $
\[
2.5c_{1} = 66 \implies c_{1} = 26.4 \quad c_{2} = 165
\]
MPC is the following:
\[
c_{1} = \frac{Y_{1}}{2.5} + 14.4 \implies c_{1}'(Y_{1}) = 0.4
\]
\item 

\end{enumerate}

\clearpage
\section*{Question 5}
\fbox{\begin{minipage}{40em}
Two individuals who are optimizing their consumption path over an infinite horizon have different
utility functions.
While the utility of individual $ u $ is given by:
\[
U(C) = \ln c_{1} + \frac{\ln c_{2}}{1+\r} + \frac{\ln c_{3}}{(1+\r)^{2}} + \dots = \sum_{t=1}^{\infty} \frac{\ln c_{t}}{(1+\r)^{t-1}}
\]
the utility of individual $ v $ is given by
\[
V(C) = \sqrt[3]{c_{1}} + \frac{\sqrt[3]{c_{2}}}{1 + \r} + \frac{\sqrt[3]{c_{3}}}{(1 + \r)^{2}} + \dots =  \sum_{t=1}^{\infty} \frac{\sqrt[3]{c_{t}}}{(1+\r)^{t-1}}
\]
Individuals $ u $ and $ v $ have the same income path $ \{y_{t}\}_{t=1}^{\infty} $. Moreover, they have the same time
preferences and face the same interest rate.\\
\begin{enumerate}[(1)]
	\item Show that both utility functions satisfy the standard assumptions assumed in our model.
	\item Find out the relationship between consumption in two subsequent periods for both individuals.
	\item Could it be that both individuals have the same consumption path? Explain
	\item Assume that the interest rate $ r $ is higher than $ \r $.
	\begin{enumerate}[(a)]
		\item How does the consumption path look like?
		\item Draw the consumption path of both individuals and compare them.
	\end{enumerate}
	\item Assume that individual $ u $ now knows that her income will increase by amount $ x $ starting from period $ 3 $.
	\begin{enumerate}[(a)]
		\item How does the consumption path of individual u change?
		\item Does you answer to the previous section depend on the whether $ r ><= p $?
		\item Draw all possible cases
	\end{enumerate}
\end{enumerate}
\end{minipage}}


\begin{enumerate}[(1)]
	\item 	Standart assumptions about the utility function:
	\begin{enumerate}[(a)]
		\item 	Non-decreasing by $ c_{t} $ and concavity with respect to $ c_{t} $. Function $ U(.) $ is concave iff $ U''(.) < 0 $:
		\[
		\frac{\partial U(C)}{\partial c_{t}} =  \frac{1}{c_{t}(1+\r)^{t-1}} > 0 \quad\quad \frac{(\partial U(C))^{2}}{\partial^{2} c_{t}} = -\frac{1}{c_{t}^{2}(1+\r)^{t-1}} < 0 \quad \forall t = 1 \dots
		\]
		\[
		\frac{\partial V(C)}{\partial c_{t}} =  \frac{1}{3\sqrt[3]{c_{t}^{2}}(1+\r)^{t-1}} > 0 \quad\quad \frac{(\partial U(C))^{2}}{\partial^{2} c_{t}} = -\frac{2}{9c_{t}^{5/3}(1+\r)^{t-1}} < 0 \quad \forall t = 1 \dots
		\]
	\end{enumerate}
	\item We need to solve the optimization problem:
	\begin{enumerate}[(a)]
		\item For agent $ u $:
		\[
		\begin{dcases}
		\sum_{t=1}^{\infty} \frac{\ln c_{t}}{(1+\r)^{t-1}} \to \max_{c_{t}} \\
			\sum_{t=1}^{\infty}  \frac{c_{t}}{\prod_{i=1}^{t-1} (1+r_{i})} = 	\sum_{t=1}^{\infty}  \frac{y_{t}}{\prod_{i=1}^{t-1} (1+r_{i})}
	\end{dcases}
		\]
		Since the task is fully equivalent to the agent's task of question 1, we can use the \textbf{EE} from there:
		\[
		\frac{u'(c_{t})}{u'(c_{t+1})} = \frac{1+r_{t}}{1+\r} \quad\implies\quad \frac{c_{t+1}}{c_{t}} = \frac{1+r_{t}}{1 + \r}
		\]
		\item For agent $ v $:
				\[
		\begin{dcases}
			\sum_{t=1}^{\infty} \frac{\sqrt[3]{c_{t}}}{(1+\r)^{t-1}} \to \max_{c_{t}} \\
			\sum_{t=1}^{\infty}  \frac{c_{t}}{\prod_{i=1}^{t-1} (1+r_{i})} = 	\sum_{t=1}^{\infty}  \frac{y_{t}}{\prod_{i=1}^{t-1} (1+r_{i})}
		\end{dcases}
		\]
		and, again, using the \textbf{EE}:
		\[
		\frac{u'(c_{t})}{u'(c_{t+1})} = \frac{1+r_{t}}{1+\r} \quad\implies\quad \frac{c_{t+1}}{c_{t}} = \sqrt[2]{\left(\frac{1+r_{t}}{1 + \r}\right)^{3}}
		\]
	\end{enumerate}
	\item Since both agents receive the sane value of income, difference in consumption is fully determined by \textbf{EE}. So, if \textbf{EE} is identical for all agents for every value of $ t $, then consumers will have the same consumption path, or mathematically:
	\begin{align*}
		\forall t=1, 2,\dots \quad\quad \frac{1+r_{t}}{1+\r}  &= \left(\frac{1+r_{t}}{1 + \r}\right)^{3/2} \\
		1 &= \left(\frac{1+r_{t}}{1 + \r}\right)^{1/2} \\
		1+\r &= 1 + r_{t}  \iff \r = r_{t}
	\end{align*}
So if both consumers face $ r_{t} = \r $, then they will have the same consumption path.
\item Assuming that $ r_{t} > \r $
\begin{enumerate}[(a)]
	\item Both eulers equations (thanks to concavity of functions $ u(c_{t}) $ and $ v(c_{t}) $) yield that next period consumption is higher than the present one, or:
	\[
	r_{t} > \r \implies c_{t+1} >c_{t}
	\]
	But for the $ u $ and $ v $ growth of consumption gain is different: $ v $'s agent consumption is increasing faster:
	\[
	\underbrace{c_{t+1} = c_{t}\frac{1+r_{t}}{1+\r}}_{\text{Agent } u} \quad\quad \underbrace{c_{t+1} = c_{t}\left(\frac{1+r_{t}}{1+\r}\right)^{3/2}}_{\text{Agent } v}
	\]
\end{enumerate}
\item
\begin{enumerate}[(a)]
	\item  Since the $ y_{t} $ does not affect the Euler Equation, the $ c_{t+1} $ to $ c_{t} $ ratio will not change. But still the absolute value for $ c_{t} $ will be increased since more income is available.
	\item yes, if $ r_{t} > \r $, then consumption is increasing.
\end{enumerate}

\end{enumerate}
\end{document}  