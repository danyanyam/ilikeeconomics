%%%%%%%%%%%%%%%%%%%%%%%%%%%%%%%%%%%%%%%%%%%%%%%%%%%%%%%%%%%%%%%%%%%%%%%%%%%%%%%%%%%%
% Do not alter this block (unless you're familiar with LaTeX
\documentclass{article}
\usepackage[margin=1in]{geometry} 
\usepackage{amsmath,amsthm,amssymb,amsfonts, fancyhdr, color, comment, graphicx, environ}
\usepackage[utf8]{inputenc}
\usepackage[russian]{babel}
\usepackage{xcolor}
\usepackage{mdframed}
\usepackage{tikz}
\usepackage{pgfplots}
\usepackage[shortlabels]{enumitem}
\usepackage{indentfirst}
\usepackage{subcaption}
\usepackage{hyperref}
\usepackage{bbm}
\usepackage{calrsfs}
\usepackage{booktabs} % Required for better horizontal rules in tables
\newcommand*\circled[1]{\tikz[baseline=(char.base)]{
		\node[shape=circle,draw,inner sep=1pt] (char) {#1};}}
\usepackage{fouriernc} % Use the New Century Schoolbook font

\usetikzlibrary{arrows}
\usepackage[most]{tcolorbox}
\usepackage{xcolor}
\usetikzlibrary{patterns}
\pgfplotsset{compat=newest}
\usepgfplotslibrary{fillbetween}
\hypersetup{
	colorlinks=true,
	linkcolor=blue,
	filecolor=magenta,      
	urlcolor=blue,
}
\definecolor{mypink1}{rgb}{0.858, 0.188, 0.478}
\newcommand\numberthis{\addtocounter{equation}{1}\tag{\theequation}}
\pagestyle{fancy}
\pgfplotsset{my style/.append style={axis x line=middle, axis y line=
		middle, xlabel={$x$}, ylabel={$y$}, axis equal }}


\newenvironment{problem}[2][Задача]
{ \begin{mdframed}[backgroundcolor=gray!20] \textbf{#1 #2.} \\}
	{  \end{mdframed}}


\newenvironment{solution}{\textbf{Решение:} \\}

%%%%%%%%%%%%%%%%%%%%%%%%%%%%%%%%%%%%%%%%%%%%%
%Fill in the appropriate information below
\lhead{Даниил Бучко, MAE'23 (B)}
\rhead{Теория вероятностей - 2021} 
\chead{\textbf{Домашняя работа №2}}
%%%%%%%%%%%%%%%%%%%%%%%%%%%%%%%%%%%%%%%%%%%%%


\begin{document}
	
\begin{problem}{1}
	
	Пусть на пространстве элементарных исходов $ \Omega $ с сигма-алгеброй событий $ \mathcal{F} $ задана конечно аддитивная вероятность $ P $, т.е. вместо аксиомы счетной аддитивности выполнена аксиома конечной аддитивности: если $ A_{1}\dots A_{n}\in \mathcal{F} $  и $ A_{1}A_{j} = \emptyset, i\ne j$ то $ P\left(\bigcup_{i=1}^{n}A_{i}\right) =  \sum_{i=1}^{n}P(A_{i})$.  Докажите, что если $ P $ непрерывна в нуле (см. лекцию 3), то она счетно аддитивна
	
\end{problem}

\begin{solution}
	Требуется доказать, что выполняется следующее равенство: 
	\[P\left(\bigcup_{i=1}^{\infty}A_{i}\right) =  \sum_{i=1}^{\infty}P(A_{i})\]
	По условию имеем, что $ \{A_{n}\} $ -- последовательность событий с пустым пересечением, а значит можно переписать бесконечное объединение событий последовательности в следующем виде:
	\[
	P\left(\bigcup_{i=1}^{\infty}A_{i}\right) =  P\left(\bigcup_{i=1}^{n}A_{i}\right) + P\left(\bigcup_{i=n+1}^{\infty}A_{i}\right)
	\]
	применяя сигма-аддитивность, и рассматривая бесконечное объединение событий от $ n+1 $ как отдельное множество, получим:
	\begin{align*}
	\lim_{n\to\infty} P\left(\bigcup_{i=1}^{n}A_{i}\right)   = \lim_{n\to\infty} \left\{P\left(\bigcup_{i=1}^{\infty}A_{i}\right) - \underbrace{P\left(\bigcup_{i=n+1}^{\infty}A_{i}\right) }_{\to 0}\right\} = \lim_{n\to\infty} P\left(\bigcup_{i=1}^{\infty}A_{i}\right) =  \sum_{i=1}^{\infty}P(A_{i})
	\end{align*}
	Поясним переходы выше. Заметим, что  $B_{n} = \bigcup_{i=n+1}^{\infty}A_{i}  $  -- последовательность непересекающихся, вложенных событий. Из этого следует по аксиоме о непрерывности, что $ P(B_{n}) $ стремится к нолю, при $ n\to\infty $.
\end{solution}
	
	

\begin{problem}{2}
	В коробке 40 конфет M\&M’s, 12 красного цвета, 8 желтого цвета, 9 коричневого
	цвета и 11 зеленого цвета. Случайным образом (неупорядоченно и без возвращения)
	выбираются 9 конфет. Найдите вероятность того, что в выборке будет
	\begin{enumerate}
		\itemsep0em 
		\item три красных конфеты
		\item две красных, три желтых, четыре зеленых конфеты
	\end{enumerate}
\end{problem}

\begin{solution}
	\begin{enumerate}
		\item 	$ A $ -- событие, при котором в выборке ровно 3 красных конфеты \\
		$ C_{40}^{9} $ -- количество способов взять 9 конфет (неупорядоченно и без возвращения) \\
		$ C_{12}^{3} $ -- количество способов выбрать 3 красных конфеты из 12 возможных \\
		$ C_{28}^{6} $ -- количество способов взять 6 НЕ красных конфет 
		\[P(A) = \frac{C_{12}^{3}C_{28}^{6}}{C_{40}^{9}} \sim 0.303 \]
		\item $ B $ -- событие, при котором в выборке ровно 2 красных, 3 желтых, 4 зеленых конфеты\\
		$ C_{40}^{9} $ -- количество способов взять 9 конфет (неупорядоченно и без возвращения) \\
		$ C_{12}^{2} $ -- количество способов взять 2 красных конфеты \\
		$ C_{8}^{3} $ -- количество способов взять 3 желтых конфеты \\
		$ C_{11}^{4} $ -- количество способов взять 4 зеленых конфеты \\
		\[ P(B) = \frac{C_{12}^{2}C_{8}^{3}C_{11}^{4}}{C_{40}^{9}} \sim 0.004 \]
		
	\end{enumerate}
	
	
\end{solution}


\begin{problem}{3}
	Есть три специальных игральных кубика: с числами 1,4,4,4,4,4 на гранях, с числами 2,2,2,5,5,5 на гранях с числами 3,3,3,3,3,6 на гранях. Два игрока, $ A $ и $ B $, играют в следующую игру: $ B$ выбирает один кубик из трех, $ A $ выбирает один кубик из двух оставшихся; затем игроки подбрасывают свои кубики, и выигрывает тот, у кого выпало больше очков. Игроки выбирают кубики не случайно, а с целью повысить вероятность выигрыша.
	\begin{enumerate}
		\itemsep0em 
		\item  Кто из игроков имеет больший шанс выиграть?
		\item  Найдите вероятность того, что выиграет $ B $.
	\end{enumerate}
\end{problem}

\begin{solution}
	Чтобы понять, как именно игроки будут выбирать кубики и есть ли у них доминирующие выигрышные стратегии, необходимо просчитать вероятность победы второго игрока в зависимости от выбранного кубика в ответ на выбор кубика первым игроком. Обозначим событием $ W_{i}^{A} $  -- победные исходы, которые приносит выбор $ i $-го кубика игроку $ A $. Кроме того: $ A_{i} $ -- номер кубика, выбранный игроком $ A $ и $ B_{i} $ -- номер кубика, выбранный игроком $ B $. Посчитаем вероятности победы второго игрока в ответ на выбор каждого из кубиков первым игроком. \\
	
	Для начала, подчеркнем очевидные вещи. Если первый игрок взял $ i $-ый кубик, то этот кубик становится недоступным для выбора второму игроку. Рассмотрим поочередно ситуации, в которых первый игрок берет каждый из кубиков.
	\begin{enumerate}
		\item Случай $ B_{I} $ (первый игрок взял кубик типа $ I $). В ответ на эту стратегию второй игрок может выбрать либо второй кубик, либо третий. Вероятность победы игрока $ A $ для каждой из стратегий соответственно равны:
		\begin{align*}
		P\left(W_{II}^{A} | B_{I}\right) = \frac{1}{6}\frac{6}{6} + \frac{5}{6}\frac{3}{6} = \frac{21}{36}\\
		P\left(W_{III}^{A} | B_{I}\right) = \frac{1}{6}\frac{6}{6} + \frac{5}{6}\frac{1}{6} = \frac{11}{36}
		\end{align*}
		Кратко поясню, откуда берутся эти числа. Рассмотрим случай $ P\left(W_{II}^{A} | B_{I}\right) $. У игрока $ B $ при бросании кубика $ I $ возможны выпадения двух чисел: либо 1, либо 4. Если выпала единица (что происходит с вероятностью $ 1/6 $), то игрок $ A $ выиграет, при условии, что на кубике $ II $ выпало число большее, чем единица (что происходит с вероятностью $ 1 $). \\ Кроме единицы у $ B $ могла выпасть $ 4 $ (с вероятностью $ 5/6 $), победить которую игроку $ A $  при помощи второго кубика можно было в 3-х из 6-х исходов. 
		\item Случай $ B_{II} $ (первый игрок взял кубик типа $ II $).  В ответ на эту стратегию второй игрок может выбрать либо первый кубик, либо третий. Вероятность победы игрока $ A $ для каждой из стратегий соответственно равны:
		\begin{align*}
		P\left(W_{I}^{A} | B_{II}\right) &= \frac{3}{6}\frac{5}{6}  = \frac{15}{36}\\
		P\left(W_{III}^{A} | B_{II}\right) &= \frac{3}{6}\frac{6}{6} + \frac{3}{6}\frac{1}{6} = \frac{21}{36}
		\end{align*}
		\item И наконец рассмотрим финальный случай $ B_{III} $ (первый игрок взял кубик типа $ III $). В ответ на эту стратегию второй игрок может выбрать либо первый кубик, либо второй. Вероятность победы игрока $ A $ для каждой из стратегий соответственно равны:
		\begin{align*}
		P\left(W_{I}^{A} | B_{III}\right) &= \frac{5}{6}\frac{5}{6}  = \frac{25}{36}\\
		P\left(W_{II}^{A} | B_{III}\right) &= \frac{5}{6}\frac{3}{6} = \frac{15}{36}
		\end{align*}
	\end{enumerate}
Аггрегируя полученные результаты, сделаем таблицу с различными вероятностями на победу игрока $ A $  в зависимости от выбора кубика игроком $ B $:
\begin{table}[hbtp!]
	\centering 
	\begin{tabular}{l c c c } 
		\toprule 
		& \multicolumn{3}{c}{\textbf{A}} \\ 
		\cmidrule(l){2-4} 
		\textbf{B} & I & II & III \\
		\midrule
		I & x & \color{mypink1}$21/36$ & $ 11/36 $\\ 
		II & $ 15/36 $ & x & \color{mypink1}$ 21/36 $ \\
		III &  $25/36 $ & $ 15/36 $ & x \\ 
		\bottomrule 
	\end{tabular}
	\caption{Вероятности выиграть игроку $ A $ в зависимости от выбранного кубика} 
	\label{table}
\end{table}

Проанализируем полученные результаты из таблицы \ref{table}. В первую очередь, заметим, что игра с кубиками -- это игра с нулевой суммой, а значит вероятности победить первому игроку могли быть получены из таблицы выше, вычитанием из единицы этих результатов. Поскольку игра последовательная, то равновесные исходы определяет второй игрок $ A $. \textcolor{mypink1}{Розовым}  цветом я выделил все возможные исходы игры, исходя из предпосылки, что агенты выбирают кубики, максимизируя вероятность собственной победы. Игрок $ B $ точно не выберет в свой ход кубик $ III $, потому что в таком случае игрок $ A $ точно выберет первый кубик. Таким образом, кубики будут выбираться либо в последовательности $ I \to II $, либо $ II \to I $. В любом из случаев, чаще побеждать будет второй игрок с вероятностью $ 21/36 $. Игрок $ B $ будет выигрывать с вероятностью $ 1 - 21/36 = 15/36 $.



\end{solution}


\begin{problem}{4}
	Вероятность того, что письмо находится в письменном столе, равна $ p $, причем с
	равной вероятностью оно может находиться в любом из восьми ящиков стола. Было
	просмотрено $ 7 $ ящиков и в них письмо не было обнаружено. Чему равна вероятность того, что письмо находится в восьмом ящике?
\end{problem}

\begin{solution}
	Пусть событие $ A $ - письмо находится в столе, событие $ B_{i} $ -- письмо находится в $ i $-ом ящике, и событие $ \bar{B_i} $ -- письмо не находится в $ i $-ом ящике. Понятно, что если письмо не находится в столе, то вероятность его нахождения в любом из ящиков равна нолю. С другой стороны, если письмо находится в столе, то вероятность его нахождения в $ i $-ом ящике составляет $ 1/8 $. Таким образом, искомая вероятность вычисляется следующим образом:
	\begin{align*}
	P\left(B_{8} | \bar{B}_{1}\dots \bar{B}_{7}\right) =  1 - P\left(\bar{B}_{8} | \bar{B}_{1}\dots \bar{B}_{7}\right) = 1 -  \frac{P(\bar{B}_{1}\dots \bar{B}_{8})}{P(\bar{B}_{1}\dots \bar{B}_{7})}\\
	1 - \frac{1-p}{p/8 + 1-p} = \frac{p/8 + 1-p - 1 + p}{p/8 + 1-p} = \frac{p}{8-7p}\\
	\end{align*}
\end{solution}


\begin{problem}{5}
	В квадрате $ Q = \{ (x,y): 0\le x,y \le 1\}  $ случайно выбирается точка с координатами $ (\xi, \eta) $. Для каждого вещественного числа $ u $ найдите вероятности $ P(G(\xi, \eta) \le u) $, если:
	\begin{enumerate}
		\item $ G(\xi, \eta) = \min\{\xi, \eta\} $
		\item $ G(\xi, \eta) = \max\{\xi, \eta\} $
		\item $ G(\xi, \eta) = \xi + \eta$
	\end{enumerate}
\end{problem}
\begin{solution}
		\begin{minipage}{0.5\textwidth}
			\underline{Случай первый}, $ G(\xi, \eta) = \min\{\xi, \eta\} $. В задаче просят посчитать вероятность $ P( \min\{\xi, \eta\} \le u) $. Геометрически, функция $ G(\xi, \eta) = u$ представляет собой множество точек, разделяющее области, соответсвующие рисунку справа. Таким образом, искомая вероятность может быть вычислена как отношени площади заштрихованной области к площади единичного квадрата, или: 
			\begin{align*}
				P( \min\{\xi, \eta\} \le u) = 1 - P( \min\{\xi, \eta\} > u) &= 1 - (1-u)^2\\
				u(2 - u) &= 2u-u^2
			\end{align*}
			
		\end{minipage}
			\begin{minipage}{0.5\textwidth}
			\begin{tikzpicture}
			\begin{axis}
			[
			width=3.5in,
			clip=false,
			axis lines=middle,
			xmin=0,
			xmax=1.3,
			ymin=0,
			ymax=1.3,
			xtick={\empty},
			ytick={\empty},
			xlabel=$x$,
			ylabel=$y$,
			xlabel style={at={(ticklabel* cs:0.98)},anchor=north west},
			ylabel style={at={(ticklabel* cs:0.98)},anchor=south west},
			]
			\addplot [name path = B,domain = 0:1,samples = 2] {1};

			\path[draw,pattern=horizontal lines] (0,0) -- (0,1) -- (0.25,1) -- (0.25,0.25) -- (1,0.25) -- (1, 0) ;
			\node at (0.35,0.12) [fill=white]{$ P( \min\{\xi, \eta\} < u)  $};
			\node at (0.65,0.65) [fill=white]{$ P( \min\{\xi, \eta\} > u)  $};
			\draw  (axis cs:{1,0}) -- (axis cs:{1,1});

			
			\draw (-0.05,1) node {$ 1 $};
			\draw (0.46,0.3) node {$ \min\{x, y\} = u $};
			\draw (1,-0.09) node {$ 1 $};
			\draw (1.05,0.25) node {$ u $};
			\draw (0.25,1.05) node {$ u $};
						
			\draw (-0.05, -0.09) node {$ 0 $};
			\end{axis}
			\label{min}
			\end{tikzpicture}
		\end{minipage}
	\begin{minipage}{0.5\textwidth}
		\begin{tikzpicture}
		\begin{axis}
		[
		width=3.5in,
		clip=false,
		axis lines=middle,
		xmin=0,
		xmax=1.3,
		ymin=0,
		ymax=1.3,
		xtick={\empty},
		ytick={\empty},
		xlabel=$x$,
		ylabel=$y$,
		xlabel style={at={(ticklabel* cs:0.98)},anchor=north west},
		ylabel style={at={(ticklabel* cs:0.98)},anchor=south west},
		]
		\addplot [name path = B,domain = 0:1,samples = 2] {1};
		
		\path[draw,pattern=horizontal lines] (0,0.7) -- (0.7,0.7) -- (0.7,0) -- (0, 0) ;
		\node at (0.35,0.22) [fill=white]{$ P( \max\{\xi, \eta\} < u)  $};
		\node at (0.72,0.85) [fill=white]{$ P( \max\{\xi, \eta\} > u)  $};
		\draw  (axis cs:{1,0}) -- (axis cs:{1,1});
		
		
		\draw (-0.05,1) node {$ 1 $};
		\draw (0.22,0.76) node {$ \max\{x, y\} = u $};
		\draw (1,-0.09) node {$ 1 $};
		\draw (-0.05,0.7) node {$ u $};
		\draw (0.7,-0.05) node {$ u $};
		
		\draw (-0.05, -0.09) node {$ 0 $};
		\end{axis}
		\end{tikzpicture}
	\end{minipage}
			\begin{minipage}{0.5\textwidth}
				\underline{Случай второй}, $  G(\xi, \eta) = \max\{\xi, \eta\} $. Аналогичным образом определяем искомую вероятность, как площадь образуемую множеством точек $ \{ (x, y)| \max\{\xi, \eta\} \le u \} $ (площадь заштрихованного квадрата)
	\[
	P( \max\{\xi, \eta\} \le u) = u^2
	\]
\end{minipage}
		\begin{minipage}{0.5\textwidth}
	\underline{Случай третий}, $ G(\xi, \eta) = \xi + \eta $. Для начала немного преобразуем искомую вероятность:
	\[
	P(\xi + \eta \le u) = P(\eta \le u - \xi)
	\]
	Из полученного преобразования становится понятным, что искомая вероятность -- площадь под функцией $ u - \xi$ (рисунок справа). Аналитически её можно представить в следующем виде:
	\[
	P(\xi + \eta \le u) = 
	\begin{cases} u^2/2, &  u \le 1 \\ 
	1 - (2-u)^2/2, & u > 1 
	\end{cases}
	\]
	Вероятность получилась кусочно-заданной, потому что при $ u\le 1 $ мы рассматриваем площадь треугольника, а при $ u > 1 $ -- площадь фигуры, напоминающей трапецию.
\end{minipage}
\begin{minipage}{0.5\textwidth}
	\begin{tikzpicture}
	\begin{axis}
	[
	width=3.5in,
	clip=false,
	axis lines=middle,
	xmin=0,
	xmax=1.3,
	ymin=0,
	ymax=1.3,
	xtick={\empty},
	ytick={\empty},
	xlabel=$x$,
	ylabel=$y$,
	xlabel style={at={(ticklabel* cs:0.98)},anchor=north west},
	ylabel style={at={(ticklabel* cs:0.98)},anchor=south west},
	]
	\addplot [name path = B,domain = 0:1,samples = 2] {1};
	
	\path[draw,pattern=horizontal lines] (0,0.8) -- (0.8,0) -- (0, 0) ;
	\node at (0.35,0.12) [fill=white]{$ P(\xi + \eta < u)  $};
	\node at (0.65,0.65) [fill=white]{$ P(\xi + \eta > u)  $};
	\draw  (axis cs:{1,0}) -- (axis cs:{1,1});
	
	
	\draw (-0.05,1) node {$ 1 $};
	\draw (0.46,0.4) node[rotate=-40]{$\xi + \eta = u $};
	\draw (1,-0.09) node {$ 1 $};
	\draw (1.05,0.25) node {$ u $};
	\draw (0.25,1.05) node {$ u $};
	
	\draw (-0.05, -0.09) node {$ 0 $};
	\end{axis}
	\end{tikzpicture}
\end{minipage}

\end{solution}

\end{document}