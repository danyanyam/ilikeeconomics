%%%%%%%%%%%%%%%%%%%%%%%%%%%%%%%%%%%%%%%%%%%%%%%%%%%%%%%%%%%%%%%%%%%%%%%%%%%%%%%%%%%%
% Do not alter this block (unless you're familiar with LaTeX
\documentclass{article}
\usepackage[margin=1in]{geometry} 
\usepackage{amsmath,amsthm,amssymb,amsfonts, fancyhdr, color, comment, graphicx, environ}
\usepackage{xcolor}
\usepackage{mdframed}
\usepackage[shortlabels]{enumitem}
\usepackage{indentfirst}
\usepackage{hyperref}
\usepackage[utf8]{inputenc}
\usepackage[russian]{babel}
\usepackage{fouriernc} % Use the New Century Schoolbook font
\hypersetup{
    colorlinks=true,
    linkcolor=blue,
    filecolor=magenta,      
    urlcolor=blue,
}


\pagestyle{fancy}


\newenvironment{problem}[2][Задача]
    { \begin{mdframed}[backgroundcolor=gray!20] \textbf{#1 #2} \\}
    {  \end{mdframed}}

% Define solution environment
\newenvironment{solution}{\textbf{Решение}}

%%%%%%%%%%%%%%%%%%%%%%%%%%%%%%%%%%%%%%%%%%%%%
%Fill in the appropriate information below
\lhead{Даниил Бучко}
\rhead{Теория вероятностей'21} 
\chead{\textbf{Домашняя работа}}
%%%%%%%%%%%%%%%%%%%%%%%%%%%%%%%%%%%%%%%%%%%%%


\begin{document}

    \begin{problem}{1}
    	Пусть $ Z\sim \mathcal{N}(0, 1) $ -- стандартная нормальная с.в. Случайная величина $ U $ имеет следующее распределение: $ P\left(U = 1\right) = 1/2 $ и $ P\left(U = -1\right) = 1/2 $ и не зависит от $ Z $.
    	
    	\begin{enumerate}
    		\item Докажите, что с.в $ X=UZ $ имеет нормальное распределение и найдите параметры этого распределения
    		\item Покажите, что распределение случайной величины $ Y = X + Z  $ не является нормальным
    	\end{enumerate}
    \end{problem}
    \begin{solution}
    
    \begin{enumerate}
    	\item Распределение $ P(X \le x) = P (UZ \le x) $ можно расписать следующим образом:
    	\begin{gather*}
    	P (UZ \le x) = P(U=1)P(UZ \le x~|~ U=1 ) + P(U=-1)P(UZ \le x~|~ U = -1) =\\ \frac{1}{2}P(UZ \le x~|~ U=1 ) + \frac{1}{2}P(UZ \le x~|~ U=-1 ) = \frac{1}{2}\left(P(Z\le x) + P(Z\ge -x)\right) = \\
    	\frac{1}{2}\left( P(Z\le x) + P(Z\le x) \right) = P(Z \le x) = \Phi(x)
    	\end{gather*}
    	Причем предпоследнее равенство верно в силу симметричности функции плотности нормального распределения. Таким образом $ X \sim \mathcal{N}(0, 1) $.
    	\item 
    	\begin{gather*}
    		P(Y \le y) = P(X + Z \le y) = P(UZ + Z \le y) = P[Z(U+1) \le y] = \\
    		P(U = 1)P(2Z\le y) + P(U=-1)P(Z * 0 \le y) = \\
    		\frac{1}{2}\left(P\left(Z\le \frac{y}{2}\right) + P(Z*0 \le y)\right) = \frac{1}{2}\left(\Phi (y/2) + I_{0 \le y}\right)
    	\end{gather*}
    	Заметим, что в ноле получили разрыв, из-за чего искомая вероятность перестает быть распределением нормальной случайной величины.
    \end{enumerate}
    \end{solution}
	\begin{problem}{2}
		Проводится $ n $ независимых испытаний. В каждом испытании возможен один из $ m $ исходов. Исход $ j $ появляется с вероятностью $ p_{j} $, эти вероятности не зависят от номера испытания. Обозначим через $ \pi_{j} $ общее число появлений исхода $ j $. Случайный вектор $ \pi = [\pi_{1}, \dots, \pi_{m} ]' $ называется полиномиальным случайным вектором.
		\begin{enumerate}
			\item Найдите распределение вектора $ \pi $
			\item Вычислите вектор средних значений $ E(\pi) $ и ковариационную матрицу $ V(\pi) $
		\end{enumerate}
	\end{problem}

\begin{solution}
	
	Заметим, что распределение $ \pi_{j} $	совпадает с распределение суммы бернулливских случайных величин, которые равны единице с вероятностью $ p_{j} $ и нолю с вероятностью $ 1 - p_{j} $. Обозначим $ \xi_{i}^{j} \sim Bern(p_{j})$, случайную величину появления $ j $-го исхода в $ i $-ом эксперименте. Тогда:
	\[
	\pi_{j} = \sum_{i=1}^{n}\xi_{i}^{j}
	\]
	Здусь $ \pi_{j} $ -- $ j $-ая компонента вектора $ \pi $. Чтобы понять, как выглядит распределение $ \pi_{j} $, рассмотрим простейший случай, когда $ i=2 $ (далее опустим верхние индексы):
	\begin{gather*}
		P(\pi_{j} = x) = P(\xi_{1} + \xi_{2} = x) = P(\xi_{2} = 1)P(\xi_{1} = x- 1) + P(\xi_{2} = 0)P(\xi_{1} = x)
	\end{gather*}
	Если $ i=3 $, то:
		\begin{align*}
	P(\pi_{j} = x) = P(\xi_{1} + \xi_{2} + \xi_{3} = x) = &P(\xi_{1} = x)P(\xi_{2} = 0, \xi_{3} = 0) +\\
	& P(\xi_{1} = x-1)[P(\xi_{2} = 0)P(\xi_{3} = 1) + P(\xi_{2} = 1)P(\xi_{3} = 0)] + \\
	&P(\xi_{1} = x - 2)P(\xi_{2} = 1, \xi_{3} = 1)
	\end{align*}
	Также заметим, что $ P(\xi_{i}^{j} = 1) = p_{j} $. Тогда случаи $ i=2 $ и $ i=3 $ могут быть упрощены до:
	\begin{enumerate}
		\item $ i = 2 $:
		\[
		(1-p_j)P(\xi_{1} = x) + p_{j}P(\xi_{1} = x -1) 
		\]
		\item $ i = 3 $:
		\[
		(1-p_{j})^{2}P(\xi_{1} = x) + P(\xi_{1} = x -1)[C^{1}_{2}p_{j}(1-p_j)] + p_{j}^{2}P(\xi_{1} = x - 2)
		\]
	\end{enumerate}
Продолжая рассуждения по индукции получим следующее выражение для $ i=n $:

\begin{align*}
	P(\pi_{j} = x) = P\left(\sum_{i=1}^{n}\xi_{i} = x\right)  = &P(\xi_{1} = x)(1-p_{j})^{n-1}  + \\ 
	&P(\xi_{1} = x -1)C^{1}_{n-1}p_{j}(1-p_{j})^{n-2} + \\
	&P(\xi_{1} = x -2)C^{2}_{n-1}p_{j}^{2}(1-p_{j})^{n-3} + \\
	& \dots  \\
	&P(\xi_{1} = x - n - 1)p_{j}^{n-1} = Q_{j}(x)
\end{align*}
Таким образом вероятность $ P(\pi = x) $ соотвествует вектору функций от $ x $ размерностью $ j\times 1 $, где по строкам -- вероятности компонент, полученные в выражении выше или иначе:
\[
P(\pi = x) = [Q_{1}(x), \dots, Q_{m}(x)]'
\]

Теперь посчитаем матожидание случайного вектора $ E(\pi) $:
\[
E(\pi) = [ E(\pi_{1}), \dots, E (\pi_{m}) ]'
\]
Матожидание $ j $-ой компоненты равно соответственно (пользуемся независимостью):
\[
E(\pi_{j}) = E \left(\sum_{i=1}^{n}\xi_{i}^{j}\right) = \sum_{i=1}^{n}E(\xi_{i}^{j}) = \sum_{i=1}^{n} p_{j}=  np_{j}
\]
Тогда $ E(\pi) $ будет равно:
\[E(\pi) = n[p_{1}~ p_{2} ~\dots~ p_{m}]'\]
Дисперсия $ j $-ой компоненты равна соответсвенно (пользуемся независимостью):
\[
D(\pi_{j}) = D \left(\sum_{i=1}^{n}\xi_{i}^{j}\right) = \sum_{i=1}^{n}D(\xi_{i}^{j}) = \sum_{i=1}^{n} p_{j}(1-p_{j})=  np_{j}(1-p_{j})
\]
Тогда $ Cov(\pi) $ будет диагональная матрица, у которой по диагонали будет вектор (а во всех других элементах -- нули):
\[
D(\pi) = n[~p_{1}(1-p_{1}) ~ \dots ~ p_{m}(1-p_{m})~]'
\]

\end{solution}

\begin{problem}{3}
	Доходности акций двух компаний являются случайными величинами $ X $ и $ Y $ с одинаковыми средними и матрицей ковариаций \[ V = \begin{bmatrix}
		6& -3\\ -3&9
	\end{bmatrix}\] В какой пропорции надо купить акции этих компаний, чтобы дисперсия доходности получившегося портфеля была
	наименьшей? Доходность портфеля считается по формуле $ R = aX + (1-a)Y $
\end{problem}
\begin{solution}
	Проведем задачку оптимизации дисперсии доходности, оптимизируя параметр $ a $:
	\begin{gather*}
		\min_{a>0} D(R) = D(aX + (1-a)Y) = a^{2}D(X) + 2a(1-a)Cov(X, Y) + (1-a)^{2}D(Y) \to \min_{a} 
	\end{gather*}
	Учитывая, что $ D(X) = 6$, $ D(Y) = 9 $, $ Cov(X, Y) = -3 $, наша оптимизационная задача превращается в
	\begin{gather*}
	6a^{2} -6a(1-a) + 9(1-a)^2 \to\min_{1\ge a\ge0}  = \\ 
	21a^{2}-21a+9 \to\min_{a}
	\end{gather*}
	минимум параболы с ветвями вверх достигается в вершине с координатой $ a = 12/21$.
\end{solution}

\begin{problem}{4}
	У неправильной монеты вероятность выпадения <<орла>> равна $ 0.3 $. Монета подбрасывается $ 400 $ раз. Пусть $ X $ -- число выпадений <<орла>>. Оцените $ P(100 \le X \le 140) $
	\begin{enumerate}
		\item Используя неравнство Чебышёва
		\item Используя нормальное приближение
	\end{enumerate}
\end{problem}
\begin{solution}
	\begin{enumerate}
		\item 
	
	По определению неравенства Чебышева:
	\[
	P(|X - E(X)|>\varepsilon) \le \frac{D(X)}{\varepsilon^{2}}, \quad \varepsilon > 0
	\]
Легко заметить, что $ E(X) = E(400\xi) = 400E(\xi) = 120$ и $ D(X)  = D(\sum_{i=1}^{400}\xi)= 400*0.3*0.7= 84$ где $ \xi  $ -- распределение бернулли с параметром $ 0.3 $. Тогда, искомую вероятность можно переписать в ином виде:
\[
P(100 \le X \le 140)  = P(|X- 120| \le 20) >1- \frac{\sigma^{2}}{20^{2}} = 0.79
\]
Ответ: вероятность того, что сумма очков на подброшенной 400 раз монете будет лежать между 100 и 400 больше $ 0.79 $.
\item 
Используем нормальное приближение:
\[
	P(100 \le X \le 140) = P\left(\frac{100 - 120}{9.1} \le Z \le \frac{100 + 120}{9.1}\right) = P(-2.19\le Z \le 24) = 
\]
\[
\Phi(24) - \Phi(-2.19) = 1 - 0.01426 = 0.98574
\]
\end{enumerate}
\end{solution}

\begin{problem}{5}
	Правильный игральный кубик подбрасывается до тех пор, пока сумма всех выпавших очков не превзойдет $ 300 $. Оцените вероятность того, что потребуется не менее
	$ 80 $ подбрасываний.
\end{problem}
	\begin{solution}
		Пусть с.в. $ X $ -- количество выпавших на кубике очков. Понятно, что $ X $ имеет дискретное распределение, причем поскольку кубик правильный, то выпадение каждой из сторон равновероятно и равно $ 1/6 $. По условию требуется найти следующую вероятность:
		\[
		E\left(\sum_{i=1}^{80} X_{i}\right)  = 80 E(X) = 80\times \frac{7}{2} = 280 \quad\quad D\left(\sum_{i=1}^{80} X_{i}\right)  = \frac{35*80}{12}
		\]
		\[
		P\left(\sum_{i=1}^{80}X_{i} \le 300 \right) = P\left( \frac{\sum_{i=1}^{80}X_{i} - 280}{\sqrt{35\times80/12}} \le \frac{300 - 280}{\sqrt{35\times80/12}}  \right) = 0.905
		\]
	\end{solution}

\begin{problem}{6}
Пусть $ U $ -- случайная величина, равномерно распределенная на отрезке $ [0, 2\pi] $, а $ Z $ -- показательная с.в с параметром $ \lambda =1 $, и величины $ U, Z $ -- независмы. Пусть $ X = \sqrt{2Z}\cos U $, $ Y = \sqrt{2Z}\sin U $. Покажите, что $ [X, Y]' $ -- стандартный нормальный вектор.

\end{problem}
\begin{solution}
	Если $ [X, Y]' -- $ стандартный нормальный вектор, тогда совместная функция плотности случайных величин $ X,Y $ будет иметь следующий вид:
	\[
	f_{X, Y} (x, y) = \frac{1}{2\pi}e^{-(x^{2}+y^{2})/2}
	\]
	Известен следующий факт:
	\[
		f_{X,Y} (x, y) = f_{Z, U}(z, u) |J(z, u)|^{-1}
	\]
	Из независимости случайных величин $ Z $ и $ U $ следует, что:
	\[
		f_{Z, U}(z, u) = f_{Z}(z) f_{U}(u)
	\]
	Рассмотрим преобразования над случайными величнами:
\[
tgU = \frac{Y}{X}  \implies U = \arctg \frac{Y}{X}  \quad\quad 2Z = X^{2} + Y^{2} \implies Z = \frac{X^{2} + Y^{2}}{2}
\]
\[
f_{U}(x, y) = \frac{1}{2\pi} \quad\quad f_{Z}(x, y) =e^{-(x^{2} + y^{2})/2}
\]
используя независимость $ U $ и $ Z $, получаем, что:
\[
f_{Z, U}(z, u) = \frac{1}{2\pi}e^{-(x^{2}+y^{2})/2}|J_{g}(z, u)|^{-1}
\]
осталось убедиться в том, что рассматриваемый якобиан равен единице:
\begin{gather*}
	|J_{g}|^{-1} = \frac{\partial g_{1}}{\partial z} \frac{\partial g_{2}}{\partial u} - \frac{\partial g_{1}}{\partial u}\frac{\partial g_{2}}{\partial z}
\end{gather*}
	где $ g_{1}(z, u) = \sqrt{2z}\cos u $, $ g_{2}(z, u) = \sqrt{2z}\sin u $
	
	\begin{gather*}
		\frac{\partial g_{1}}{\partial z} \frac{\partial g_{2}}{\partial u} - \frac{\partial g_{1}}{\partial u}\frac{\partial g_{2}}{\partial z} = \frac{\cos u}{\sqrt{2z}} \sqrt{2z} \cos u + \frac{\sin  u}{\sqrt{2z}} \sqrt{2z} \sin u = \sin^{2}u + \cos^{2} u = 1
	\end{gather*}
	Вывод: совместная функция плотности случайных величин $ X $ и $ Y $ имеет нормальное распределение, а значит и сами $ X $ и $ Y $ имеют нормальное распределение.
\end{solution}



\end{document}
