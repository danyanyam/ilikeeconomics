
\documentclass[12pt,reqno]{amsart}

\usepackage{graphicx}
\usepackage{mathtools}
\usepackage{amssymb}
\usepackage{enumerate}
\usepackage[shortlabels]{enumitem}
\usepackage{amsthm}
\theoremstyle{plain}
\renewcommand*{\proofname}{Solution}
\newcommand{\E}{\mathbb{E}}
\newcommand{\Ve}{V^{e}}
\newcommand{\Vu}{V^{u}}
\newtheorem*{theorem*}{Question}
%% this allows for theorems which are not automatically numbered
\usepackage{lineno}

%% The above lines are for formatting.  In general, you will not want to change these.
\title{Home Assignment 3}
\author{Daniil Buchko}
\begin{document}
\maketitle

\begin{theorem*}[1]
    \normalfont
    An unemployed agent is looking for a job. Regardless of whether the agent is employed or not,
    wage offers on the market follow a Markov process $ x_t $ that is governed by a Markov chain
    with the transition matrix
    \[
        P = \begin{pmatrix}
            0.5 & 0   & 0.5 \\
            0.9 & 0.1 & 0   \\
            0.2 & 0   & 0.8
        \end{pmatrix}
    \]
    and numeric values of the wages are given by the vector $ W = \left( 15 ~ 12 ~ 2\right)^{T} $.
    If the agent is unemployed and accepts the current offer, he earns that wage starting from the
    current period and until he departs.
    \\\\
    The employer cannot terminate the worker; however, the worker can quit with a one-period
    advance notice. That is, in any period $ t $ in which he has a job, after learning $ x_t $ and
    getting paid for the current period, the worker may give his employer a notice, and in this case
    he enters the job market in period $ t+1 $ -- draws an offer and decides whether to accept it or
    stay unemployed and try his luck in the period/periods. The agent’s objective is to maximize
    $ \E \left[ \sum_{t=0}^{\infty} \beta^{t}y_{t} \right] $, where $ y_{t} $ is her income -- either
    $ \omega $ if she is employed with the wage $ \omega $, or $ c $ if she is uemployed in period
    $ t $.  The discount factor $ \beta=4/5 $ and the unemployment compensation $ c = 3 $.
    \begin{enumerate}
        \item Write the problem recursively (the Bellman equation). Be careful and detailed in how
              you define your value function -- ambiguity and omissions at this stage will be
              explicitly penalized.
        \item Find the optimal strategy of the worker and expected discounted present value of his
              income stream as functions of the state of the Markov process.
    \end{enumerate}
\end{theorem*}

\begin{proof}
    Let $ V_{i} $, $ V_{ij}^{e} $ and $ \Vu_{i} $ be components of the agent’s value function
    defined as follows: each of these scalar variables is the maximum of agent’s objective function
    at the beginning of the current period if he observes the curret state $ i\in\{1, 2, 3\} $ and:
    \begin{enumerate}
        \item $ V_i $ -- the agent is about to decide whether to accept the offer (continue working)
              or reject the wage offer (leave the job) at wage $ W_{i} $
        \item  $ \Ve_{ij} $-- the agent is employed at wage $ W_{j} $
        \item $ \Vu_{i} $ -- agent has just rejected the offer $ W_j $
    \end{enumerate}
    Let $ V = \left( V_1 ~ V_{2}~V_{3} \right)^{T} $ and
    $\Ve_{j} = \left( \Ve_{1j}~\Ve_{2j}~\Ve_{3j} \right)^{T}$. Then $ V_{i} $, $ \Ve_{ij} $ and
    $ \Vu_{i} $ should satisfy the following Bellman equation system:
    \[
        \begin{dcases}
            V_i = \max \{\Ve_{ii}, \Vu_{i}\}                                           \\
            \Ve_{ij} = \max \{ W_{j} + \beta P_{i}V^{e}_{j}, ~W_{j} + \beta P_{i}V  \} \\
            \Vu_i = c + \beta P_{i}V
        \end{dcases}
    \]
    Firstly lets notice that, because $ W_{1} = 15 $ is the highest salary possible, than at state
    $x_{t} = 1$ agent will choose employment:
    \begin{gather}
        V_1 = \max \{ V_{11}^{e}, V_{1}^{u} \} \implies V_{1} = \frac{15}{1 - 4/5} = 75
    \end{gather}
    By the same logic since $ 2 =: W_{3} < c := 3 $ we can state that $ V_3 = V_{3}^{u} $ and as a
    result:
    \[
        V_3 = \max \{V_{33}^{e}, V_{3}^{u} \} = V_{3}^{u} = c + \beta \left( \sum_{i=1}^{3}P_{3i}V_i \right)
        = c + \beta \left( 0.2V_{1} + 0.8 V_{3} \right)
    \]
    Therefore we can derive the value for $ V_{3} $ numerically:
    \[
        V_3 = c + \beta \left( 0.2V_{1} + 0.8 V_{3} \right) =  \frac{1}{1 - 0.8\beta} \left( c + 0.2\beta V_1  \right)
    \]
    Substituting values for parameters and $ V_{1} $:
    \begin{gather}
        V_3 = \frac{1}{0.36}\left( 3 + 12  \right) \sim 41.66
    \end{gather}
    The last thing to understand is the value for $ V_2 $:
    \[
        \begin{dcases*}
            V_2 = \max \{ V_{22}^{e}, V_{2}^{u} \} \\
            V_{22}^{e} = \max \{ W_{2} + \beta P_{2}V^{e}_{2}, ~W_{2} + \beta P_{2}V  \} \\
            V_{2}^{u} = c + \beta P_{2}V
        \end{dcases*}
    \]
    Incorporating everything to the first equation yields:
    \begin{gather*}
        V_2 = \max\{V_{22}^{e}, V_{2}^{u} \} = \max\{\max \{ \underbrace{W_{2} + \beta P_{2}V^{e}_{2}}_{A}, ~\underbrace{W_{2} + \beta P_{2}V  \}}_{B} , \underbrace{c + \beta P_{2}V}_{C} \}
    \end{gather*}
    Lets calculate every value separately:
    \begin{enumerate}[(A)]
        \item
              \[
                  W_{2} + \beta P_{2}V^{e}_{2} = 12 + \frac{4}{5} \left( \sum_{i=1}^{3}P_{2i}V_{i2}^{e}
                  \right) = 12 + \frac{4}{5} \left(  0.9 V_{12}^{e} + 0.1 V_{22}^{e} \right)
              \]
        \item
              \[
                  W_{2} + \beta P_{2}V = 12 + \frac{4}{5}\left( 0.9\underbrace{V_{1}}_{75} + 0.1V_2 \right) =
                  12 + \frac{4}{5}\left( 67.5 + 0.1 V_2 \right)
              \]
        \item
              \begin{gather*}
                  c + \beta P_{2}V = 3 + \frac{4}{5} \left( \sum_{i=1}^{3}P_{2i}V_{i} \right) = 3 +
                  \frac{4}{5} \left( 0.9 V_1 + 0.1 V_2 \right)\\ = 3 + \frac{4}{5}\left( 67.5 + 0.1V_2 \right)
              \end{gather*}
    \end{enumerate}
    Lets assume first that $ B > A $, then:
    \begin{gather}
        V_2 = B = 12 + \frac{4}{5} \left( 67.5 + 0.1V_2\right) \implies V_2 = 71.7
    \end{gather}
    If we assumed the opposite, namely $ B < A $ then:
    \[V_2 = 12 + \frac{4}{5} \left(  0.9 V_{12}^{e} + 0.1 V_{22}^{e} \right)\]


\end{proof}


\end{document}